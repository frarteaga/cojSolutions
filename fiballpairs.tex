\documentclass[a4paper,10pt]{article}
\usepackage[linkbordercolor={1 1 1},citebordercolor={1 1 1},urlbordercolor={1 1 1}]{hyperref}
\usepackage[utf8]{inputenc}

%opening
\title{Fibonacci All Pairs (COJ 2828)}
\author{Frank Arteaga Salgado\\\texttt{\small farteaga@ipvce.lt.rimed.cu}}

\begin{document}

\maketitle

\section{Problema}

Vea este problema en el Juez Caribe\~no en L\'inea, \href{http://coj.uci.cu/24h/problem.xhtml?pid=2828}{2828 - Fibonacci All Pairs}.
Pero antes de enfrentarse a este problema conviene resolver primero los problemas \href{http://coj.uci.cu/24h/problem.xhtml?pid=1596}{1596 - Fibonacci Numbers} 
y \href{http://coj.uci.cu/24h/problem.xhtml?pid=1872}{1872 - Cousin Party}.

\subsection{Enunciado}
La Secuencia de Fibonacci se define recursivamente como:

\begin{equation}
%$$
    \begin{array}{r@{ = }l}
     F_1 & 1\\
     F_2 & 2\\
     F_n & F_{n - 1} + F_{n - 2} \qquad \qquad n > 2 \\
    \end{array}
    \label{fibdef}
%$$
\end{equation}

Considere todos los pares ordenados de valores distintos de los $k$ primeros t\'erminos de la 
Secuencia de Fibonacci, esto es, $(F_i, F_j)$ con $1 \le i < j \le k$. Se desea calcular
la suma de los productos de cada uno de estos pares:

%\begin{equation}
$$
 S(k) = \sum_{1 \le i < j \le k} F_i \times F_j
$$
%\end{equation}

Ejemplo, para $k=4$ tenemos $F_1=1, F_2=2, F_3=3$ y $F_4=5$; los productos de los pares ordenados son 
$F_1 \times F_2 = 2, F_1 \times F_3 = 3, F_1 \times F_4 = 5, F_2 \times F_3 = 6, F_2 \times F_4 = 10, F_3 \times F_4 = 15$ y su suma es 
$2+3+5+6+10+15=41$.
 
\section{Soluci\'on}

Si implementamos dos ciclos anidados como sugiere el enunciado
del problema entonces obtendr\'iamos una soluci\'on $\Theta(k^2)$, la cual
es muy costosa por los rangos de $k$\footnote{$k \le 10^9$}.

Notemos que si sumamos cada par dos veces y a\~nadi\'eramos $\sum_{i=1}^{k} F^2_i$ entonces nos quedar\'ia:

\begin{equation}
 2S(k) + \sum_{i=1}^{k} F^2_i = \sum_{1 \le i, j \le k} F_i \times F_j = \left(\sum_{i=1}^k F_i\right)^2
\end{equation}
De donde se despeja $S(k)$:

\begin{equation}
 S(k) = \frac{ \left(\sum_{i=1}^k F_i\right)^2 - \sum_{i=1}^{k} F^2_i } {2}
 \label{formula}
\end{equation}
Hasta aqu\'i la expresi\'on (\ref{formula}) nos permite una implementaci\'on con costo lineal $\Theta(k)$, pero todav\'ia no
es suficiente para los rangos del problema. Por lo que es necesario encontrar expresiones para las dos sumatorias del
numerador que tengan algoritmos m\'as eficientes que el lineal para computarlas.

Esas expresiones son: 

\begin{equation}
 \sum_{i=1}^k F_i = F_{k + 2} - 2
 \label{fibac}
\end{equation}

\begin{equation}
 \sum_{i=1}^k F_i^2 = F_k \times F_{k + 1} - 1
 \label{fib2ac}
\end{equation}

Las cuales el lector puede demostrar f\'acilmente por inducci\'on. Luego de esto sustituimos (\ref{fibac}) y (\ref{fib2ac})
en (\ref{formula}):

\begin{equation}
 S(k) = \frac{ (F_{k + 2} - 2)^2 - F_k\times F_{k+1} + 1 }{2}
\end{equation}

Lo que redujo el c\'omputo de $S(k)$ al c\'alculo de tres n\'umeros de Fibonacci. Para calcular $F_n$ eficientemente
en tiempo $O(log\; n)$ se usa multiplicaci\'on de matrices explotando el siguiente hecho\footnote{Igual que las identidades (\ref{fibac}) y (\ref{fib2ac})
le recomendamos al lector demuestre (\ref{fibmat}) por inducci\'on.}

\begin{equation}
  \left( \begin{array}{r l}
     1 & 1\\
     1 & 0\\
    \end{array}  \right)^n 
    = 
    \left( \begin{array}{r l}
     F_{n + 1} & F_{n}\\
     F_{n} & F_{n - 1}\\
    \end{array}  \right)    
    \label{fibmat}
\end{equation}
cuando $F_0 = 0$ y $F_1 = 1$, que son condiciones iniciales distintas a la definici\'on (\ref{fibdef}), sin embargo 
emerge la misma secuencia de n\'umeros, pero desplazada como se muestra en la tabla:
$$
$$
\qquad\qquad\qquad\qquad
\begin{tabular}[h]{|c|c|c|c|c|c|c|c|}
\hline
 $F_0$ & $F_1$ & $F_2$ & $F_3$ & $F_4$ & $F_5$ & $F_6$ & $F_7$\\\hline
 1   & 1   & 2   & 3   & 5   & 8   & 13  & 21 \\\hline
 0   & 1   & 1   & 2   & 3   & 5   & 8   & 13 \\\hline
\end{tabular}


\end{document}
